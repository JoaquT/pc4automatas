\documentclass[12pt]{article}
\usepackage[spanish]{babel}
\usepackage[utf8]{inputenc}
\usepackage[T1]{fontenc}
\usepackage{amsmath, amssymb}
\usepackage{graphicx}
\usepackage{enumitem}
\usepackage{geometry}
\usepackage{tabularx}
\usepackage{ragged2e}
\usepackage{fancyhdr}


\usepackage{xcolor}

\usepackage{hyperref}
\usepackage{listings}
\usepackage{listingsutf8}
\usepackage{caption}

\lstset{
  inputencoding=utf8,
  extendedchars=true
}
% Definir estilo para listings de Python con fondo negro
\definecolor{bg}{RGB}{20,20,20}
\definecolor{lightgray}{RGB}{220,220,220}
\lstdefinestyle{pyblack}{
  language=Python,
  backgroundcolor=\color{bg},
  basicstyle=\ttfamily\color{lightgray}\scriptsize,
  keywordstyle=\color{cyan},
  stringstyle=\color{orange},
  commentstyle=\color{gray},
  showstringspaces=false,
  frame=single,
  rulecolor=\color{black},
  xleftmargin=0.5em,
  xrightmargin=0.5em
}

\lstdefinestyle{consoleblack}{
    language={}, % no syntax highlighting
    backgroundcolor=\color{bg},
    basicstyle=\ttfamily\color{lightgray}\scriptsize,
    showstringspaces=false,
    showtabs=false,
    showspaces=false,
    breaklines=true,
    breakatwhitespace=false,
    frame=single,
    rulecolor=\color{black},
    xleftmargin=0.5em,
    xrightmargin=0.5em,
    aboveskip=0.6em,
    belowskip=0.6em,
    columns=fullflexible,
    keepspaces=true,
    upquote=true
    inputencoding=utf8,
    extendedchars=true
    }



%Definición de márgenes y encabezados y pies de página
\geometry{a4paper, margin=2.5cm}
\setlength{\headheight}{14.61858pt}

% Definición de comandos personalizados
\newcommand\NomFacu{FACULTAD DE CIENCIAS}
\newcommand\CodCurso{CC321}
\newcommand\NomCurso{TEORÍA DE AUTÓMATAS, LENGUAJES Y COMPUTACIÓN}
\newcommand\NomEscuela{ESCUELA DE CIENCIAS DE LA COMPUTACIÓN}
\newcommand\NumPractica{4}
\newcommand\NomTrabajo{Practica \NumPractica}
\newcommand\NomCompTrabajo{Implementación de Gramáticas y Generador Pseudoaleatorio}

\newcommand\NomSeccion{A}
\newcommand\DiayHora{viernes 21 de noviembre, 2:00 p.m.}
\newcommand\ProfesorUno{- MELCHOR ESPINOZA, Victor Andrés}
\newcommand\ProfesorDos{}
\newcommand\FechaEntrega{miércoles 26 de noviembre del 2025}
\newcommand\NomAlUno{DELGADO ROMERO, Gustavo}
\newcommand\CodAlUno{20235009B}
\newcommand\NomAlDos{TORRES REATEGUI, Joaquín}
\newcommand\CodAlDos{20212661E}
\newcommand\NomAlTres{}
\newcommand\CodAlTres{}
\newcommand\CicloActual{2025-II}

\graphicspath{ {imagenes/} } % Directorio de imágenes

% Configuración de encabezados y pies de página
\fancyhead{}
\fancyfoot{}
\fancyhead[R]{\CicloActual}
\fancyhead[L]{\CodCurso - \NomCurso}
\fancyfoot[R]{\thepage}
\fancyfoot[L]{\NomFacu}
\renewcommand{\headrulewidth}{0.5pt}
\renewcommand{\footrulewidth}{0.5pt}


\begin{document}
%Título
\begin{titlepage}
    \centering
        {\bfseries\LARGE \textbf{UNIVERSIDAD NACIONAL DE INGENIERÍA} \par}
        \vspace{0.5cm}
        {\scshape\Large \NomFacu \par}
        \vspace{0.5cm}
        {\scshape\Large \NomCurso \hspace{2cm} \CodCurso \par}
        \vspace{0.5cm}
            \begin{figure}[h]
                \includegraphics[width=5cm]{LogoUNI.png}
                \centering
            \end{figure}
        {\scshape\Large \NomEscuela \par}
        \vspace{0.5cm}
        {\scshape\Large \NomTrabajo \par}
        {\scshape\LARGE 
            \textbf{\NomCompTrabajo} \par}
        \vspace{0.5cm}
            \begin{center}
                \begin{tabularx}{\textwidth} { 
                >{\raggedright\arraybackslash}m{3cm} 
                >{\raggedleft\arraybackslash}X  }
                {\Large  \textbf{Sección:} \normalsize \NomSeccion } &  {\Large  \textbf{Día y Hora:} \normalsize \DiayHora }  \\
            \end{tabularx}
        \end{center}
    
        {\Large \RaggedRight \hspace{0.05cm}  
            \textbf{Número de práctica:} \NumPractica \par}
    \vfill
        \begin{center}
            \begin{tabularx}{0.9\textwidth} { 
                 >{\raggedright\arraybackslash}X 
                 >{\raggedleft\arraybackslash}X  }
                \textbf{Apellidos y Nombres} & \textbf{Código de alumno} \\ [0.5ex] 
                \hline\hline
                \NomAlUno & \CodAlUno\\ 
                \NomAlDos & \CodAlDos\\ 
                [0.5ex] 
                \hline
            \end{tabularx}
        \end{center}
    \vfill
    {\normalsize \RaggedRight \hspace{0.05cm}  
        \textbf{Nombre de los Docentes:} \\
        \hspace{0.05cm} -\ProfesorUno \\
        \hspace{0.05cm} -\ProfesorDos
        \par}
    \vspace{0.5cm}
    {\normalsize \RaggedRight \hspace{0.05cm}  
        \textbf{Fecha de entrega del informe:} \FechaEntrega \par}
    \vfill
    {\Large \CicloActual \par}
\end{titlepage}
\pagestyle{fancy}

% Introducción
\section*{Introducción}
Este informe documenta la implementación de una gramática libre de contexto para generar historias simples y un generador de enteros pseudoaleatorios basado en el algoritmo Park--Miller. Se incluyen descripciones de diseño, fragmentos de código relevantes y ejemplos de salida.
\newpage

\tableofcontents
\newpage
\section{Especificación}
La gramática utilizada (símbolos no terminales entre \texttt{< >}):
\begin{itemize}
  \item \texttt{<inicio> -> <historia>}
  \item \texttt{<historia> -> <frase> | <frase> y <historia> | <frase> sino <historia>}
  \item \texttt{<frase> -> <articulo> <sustantivo> <verbo> <articulo> <sustantivo>}
  \item \texttt{<articulo> -> el | la | al}
  \item \texttt{<sustantivo> -> gato | niño | perro | niña}
  \item \texttt{<verbo> -> perseguia | besaba}
\end{itemize}

\section{Clase Aleatorio (Park--Miller)}
La implementación sigue el multiplicador 16807 y módulo $2^{31}-1=2147483647$. Se proporciona un método para generar el siguiente valor y otro para obtener un entero dentro de un límite.

\subsection*{Fragmento de \texttt{pseudointegers.py}}

\lstinputlisting[firstline=1, lastline=10,style=pyblack]{pseudointegers.py}

\section{Clases Regla y Gramatica}
La clase \texttt{Regla} almacena \texttt{left}, \texttt{right} (tupla de símbolos) y \texttt{cont} (contador, iniciado en 1). El método \texttt{\_\_repr\_\_} devuelve una representación textual del tipo "C L -> R1 R2 ...".

La clase \texttt{Gramatica} mantiene un diccionario de listas de reglas por símbolo izquierdo y un generador pseudoaleatorio. La selección de regla hace lo siguiente:
\begin{enumerate}
    \item Suma los contadores de las reglas disponibles (\texttt{total}).
    \item Pide un índice aleatorio en el rango $[0,\; total-1]$ \\ mediante \texttt{siguiente\_entero(total)}.
    \item Recorre las reglas restando su \texttt{cont} del índice hasta que el índice sea $\le 0$; la regla en ese punto es la elegida.
    \item Incrementa en 1 el \texttt{cont} de todas las reglas NO elegidas (comportamiento intencional documentado en este informe).
\end{enumerate}

\subsection*{Fragmentos de \texttt{gramatica.py}}
\lstinputlisting[firstline=2, lastline=58,style=pyblack]{gramatica.py}

\section{Comportamiento de los contadores}
El diseño incrementa los contadores de las reglas no elegidas, lo que favorece la selección de alternativas diferentes en iteraciones sucesivas: esta es una elección de política deliberada para diversificar las producciones generadas.

\section{Ejemplo de salida (ejecutando \texttt{probar\_gramatica()})}
Debido a la semilla fija (\texttt{seed=12345}), la ejecución típica genera tres historias aleatorias y muestra los contadores finales de \texttt{<articulo>}. Un ejemplo (ilustrativo) sería:

\lstinputlisting[firstline=1, lastline=12, style=consoleblack]{salida.txt}

\section{Observaciones y recomendaciones}
\begin{itemize}
  \item La semilla no está protegida contra el valor 0; si se desea, validar que la semilla esté en $[1,2^{31}-2]$.
  \item La función \texttt{generar} añade un espacio tras cada terminal; puede recortarse el resultado final con \texttt{strip()} para evitar espacio final.
\end{itemize}

\section*{Conclusión}
Se implementó una versión funcional de la gramática y del generador Park--Miller; el diseño de actualización de contadores se documentó como intencional para favorecer la diversidad de producciones. El código incluido es suficiente para replicar las pruebas de la práctica.

\end{document}